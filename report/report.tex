%        File: report.tex
%
\documentclass[a4paper]{article}
\usepackage[pdftex]{graphicx}
\usepackage{float}
\usepackage{hyperref}
\floatstyle{ruled}
\newfloat{listing}{hbtH}{lop}
\floatname{listing}{Listing}
\renewcommand{\topfraction}{0.85}   % This sets the percentage for how much floats get from the ‘top’ of the text of a page
\renewcommand{\textfraction}{0}   % This sets a similar percentage for how much of a page needs to be text before no more floats can be placed on that page
\renewcommand{\floatpagefraction}{0.80} % This sets how much of the page must be taken by a float before that page can be ‘all’ floats

\author{Etienne Kneuss\\
\texttt{etienne.kneuss@epfl.ch}
}
\title{Static analysis for the PHP language}
\bibliographystyle{unsrt}
\begin{document}
\maketitle
\begin{abstract}
  This report presents the work that was done to implement a static analysis
  tool for the PHP programming language. This tool aims at providing feedback
  to a developer by checking for multiple mis-uses of the language and should
  reduce the risk of encountering fatal errors at runtime after deployement of
  PHP based application.
\end{abstract}
\section{Introduction}
The PHP programming language is a compiled language. However, the compilation is
done, by default, each time a certain php file is requested. This means that the
compilation has to be fast and hence can't do much analysis and checks. Most of
the checks are thus made at runtime. The purpose of this tool is to allow the
developer to check its work for runtime-errors, without actually running its
application. Those checks are mostly based on structural checks, coupled with
a stricter type-system.
\section{Design}
This tool consists of a custom PHP compiler, translating the code
into structures such as Abstract Syntax Trees (AST) and then Control Flow Graphs
(CFG). Fixpoint analysis is then performed on the CFGs. This compiler is
divided into six distinct parts:
\begin{enumerate}
  \item Lexing
  \item Parsing and Abstract Syntax Trees
  \item Semantic Analysis
  \item Crontrol Flow Graphs
  \item Type flow Analysis
  \item Helpers
\end{enumerate}

\subsection{Lexing}
The lexer uses a modified version of JFlex\footnote{JFlex: http://jflex.de}. It
is thus written in java. The original PHP compiler uses flex as scanner. The
modification done to JFlex is the implementation of flex's \verb=yymore()=,
which currently lacks in JFlex. This function is used to tell the scanner that
the following rules should be appended to \verb=yytext= instead of overwriting
it. The tokens generated are closely related to those PHP 5.3.0 would generate.
One notable difference is that we translated every single-char tokens to their
respective name.

\subsection{Parsing and Abstract Syntax Trees}
The parser, as mentionned before, is based on a modified version of CUP. CUP is
a java based parser generator which uses a syntax similar to yacc, the parser
generator used in the original PHP compiler. The version of CUP used for this
project differs from the original one in two points:
\begin{enumerate}
  \item The original CUP cannot handle large syntax files. Indeed, since CUP
    directly writes the parsing tables as properties of the generated class,
    the size of the class properties can become too large for \emph{javacc}. The
    PHP grammar being quite big, the resulting parser class couldn't be compiled.
    To solve the problem, CUP was modified to provide an option specifying that the
    the multiple tables should be written into files and loaded at runtime. This
    modification was proposed to be included inside CUP.
  \item To avoid the troubles of referencing scala classes, the parser
    generator was modified to translate the application of the production into a
    simple tree which could then be translated by a Scala class entirely.
\end{enumerate}

The Syntax Tree (ST) is then translated into an AST. This is done by visiting
the tree and decomposing syntax elements:

\begin{listing}
\begin{verbatim}
    // ...
    def extends_from(n: ParseNode): Option[StaticClassRef] = {
        childrenNames(n) match {
            case List() =>
                None
            case List("T_EXTENDS", "fully_qualified_class_name") =>
                Some(fully_qualified_class_name(child(n, 1)))
        }
    }
    // ...
\end{verbatim}
  \caption{parser/STToAST.scala: Example of ST to AST decomposition}
\end{listing}

\begin{itemize}
  \item \verb=chidldrenNames(node)= is used to return a list of strings,
    representing the tokens or productions.
  \item \verb=child(node, index)= is used to access the N'th child
\end{itemize}

%\begin{figure}
%  \begin{center}
%  \includegraphics[scale=0.3]{ST.jpg}
%  \end{center}
%  \caption{Syntax Tree for "\$a = 1 + \$b;"}
%\end{figure}
%
%\begin{figure}
%  \begin{center}
%  \includegraphics[scale=0.3]{AST.jpg}
%  \end{center}
%  \caption{Abstract Syntax Tree for "\$a = 1 + \$b;"}
%\end{figure}

\subsection{Semantic Analysis}
Some checks can be done direcly by looking at the AST. Indeed, PHP provides a
wide range of features, some of them being considered nowadays as bad practice.
This tool emits notices for the following issues:

\begin{itemize}
  \item Non top-level declaration: PHP allows a developer to conditionnaly
    declare functions or classes. This not only generates problems with the
    following analysis, but also induces some performance hits on servers equipied
    with so-called opcode cachers. Those opcode cachers are responsible for caching
    the intermediate--or compiled version of each file, function and class. The
    goal being to speed up the process by reducing the number of compilations
    required per request. This cannot be done easily if those declarations are
    conditional.
  \item Call-time pass-by-ref: a function accepting a reference\footnote{PHP
    References: https://php.net/references} has to be defined as such, but PHP
    allows the developer to pass a reference at the time of the call, even to a
    function not declared to receive one. This feature is deprecated, and can cause
    unwanted side-effects.
  \item Non-trivial include calls: the \verb=include=\footnote{PHP Include:
    http://php.net/include} statement allows a developer to execute the given file
    in the current scope. The argument representing the file to include can be
    dynamic. This tool will try to resolve dynamic expressions that are most
    commonly used to be able to extend the analysis to that file. In case the
    expression is too dynamic, that include call will be ignored. Any
    combination of those expressions are considered as trivial dynamic expressions
    for include calls:
    \begin{itemize}
      \item the concatenation operator: "."
      \item the \verb=dirname()=\footnote{PHP dirname: http://php.net/dirname} function,
          used to retrieve the parent directory of the path passed as argument
      \item constants
      \item class constants
      \item string
    \end{itemize}
  \item Dynamic object properties: PHP allows dynamic references to an object property
    using a variable or expression (e.g. \verb.$name = "a"; $obj->$name. instead of
    \verb.$obj->a.). This is usually considered as bad practice since arrays are
    usually prefered for such tasks.
  \item Dynamic variables: PHP allows to reference a variable using either a
    variable, or an expression\footnote{PHP Variable variables:
    http://php.net/variables.variable}: (\verb=$$var= or
    \verb=${'prefix'.$name.foo()}=).
  \item Assignations in conditional expressions: assignations in PHP return the
    value assigned, they are hence valid expressions inside conditional
    expressions. However, history tells us that, most of the time, this is an
    actual typographic error replacing the assignation operator \verb&=& with
    the comparison operator \verb&==&. This tool will thus emit a warning if
    such expression is found inside an \verb&if()& or \verb&for()& condition.
    We exclude \verb&while()& on purpose as there is a common use-case where
    assignations are done directly inside the \verb&while()&
    expression\footnote{PHP mysql\_fetch\_assoc:
    http://php.net/mysql\_fetch\_assoc}.
\end{itemize}

Another part of the semantic analysis consists of validating the usage of
identifiers such as variables, functions or classes. The goal is to ensure that
no obvious semantic errors such as inheritance cycles, or visibility
inconsistances exist.

\begin{listing}
  \begin{verbatim}
<?php
class A extends D { } 
class B extends A { }
class C extends A { }
?>
  \end{verbatim}
  \caption{Class inheritance cycle}
\end{listing}

\begin{listing}
  \begin{verbatim}
test.php:2  Error: Classes A -> B -> C form an inheritance cycle
class A extends B {}
^
test.php:4  Error: Class C extends non-existent class A
class C extends A {}
                ^
2 errors occured, abort
  \end{verbatim}
  \caption{Cycle detection error}
\end{listing}

\begin{listing}
  \begin{verbatim}
<?php
class A {
    public function test() {}
}

class B extends A {
    protected function test() {}
}
?>
  \end{verbatim}
  \caption{Visibility inconsistance}
\end{listing}

\begin{listing}
  \begin{verbatim}
test.php:7  Error: Method B::test cannot overwrite A::test 
                   with visibility protected (was public)
    protected function test() {}
                       ^
1 error occured, abort
  \end{verbatim}
  \caption{Visibility error}
\end{listing}

Since PHP allows a developer to dynamically create and reference variables, it
is close to impossible to detect undeclared or unused variables solely based on
symbols collection. PHP is also loosely typed, which means type propagations
and checks cannot sanely be done using the AST only. In fact, the AST is only the
intermediate step that only let us do trivial checks. To analyse more complex
aspects of a program, this tool will use control flow graphs (CFG) which are
derived from ASTs.

\subsection{Control flow graphs}
Control flow graphs modelize the flow of values through the different control
structures. It will be used here to detect unreachable branches of code, and to
propagate types. Some basic type checks can be done, but since PHP does
automatic type juggling, only few type errors can result in fatal errors.
Even though the type error might be fatal, the purpose of this tool is to indicate
any potential mistakes, since some valid PHP code may still contain programming
mistakes.
\begin{listing}
  \begin{verbatim}
<?php
function test() {
  $a = 0;
  for($i = 0; $i < 10; $i++) {
    $a += $i;
  }
}
?>
  \end{verbatim}
  \caption{Control structure example}
  \label{cfg1}
\end{listing}

%\begin{figure}
%  \begin{center}
%  \includegraphics[scale=0.3]{CFG1.jpg}
%  \end{center}
%  \caption{Control flow graph for Listing~\ref{cfg1}}
%\end{figure}

\begin{listing}
  \begin{verbatim}
<?php
function test() {
  $a = 0;
  for($i = 0; $i < 10; $i++) {
    break;
    $a += $i;
  }
}
?>
  \end{verbatim}
  \caption{Control structure with unreachable code}
  \label{cfg2}
\end{listing}

%\begin{figure}
%  \begin{center}
%  \includegraphics[scale=0.3]{CFG2.jpg}
%  \end{center}
%  \caption{Control flow graph for Listing~\ref{cfg2}, showing unreachable branches}
%\end{figure}

\subsection{PHP overview}
Before trying to see how to analyse PHP code, it is important to understand how PHP works,
this is a small overview of the features PHP provides that are relevant to this analysis.
\subsubsection{Types}
In PHP, we have the following types:
\begin{itemize}
  \item \textbf{Booleans}
  \item \textbf{Integers}
  \item \textbf{Floating point numbers}
  \item \textbf{Strings}
  \item \textbf{Arrays}: Arrays in PHP are ordered hashmaps, allowing either strings or
    integers as keys.  They can contain values of any type that may be mixed.
  \item \textbf{Objects}
  \item \textbf{Resources}: Resources represent special data types such as
    file handle, database conection links. They can be of different types. PHP
    extensions that define resource types are responsible of handling them
    appropriately.
  \item \textbf{Null}: This is the default type to any undefined or
    uninitialized variable.
\end{itemize}

Because PHP is loosely typed, it will do implicit type conversions\footnote{PHP
Type Juggling: php.net/language.types.type-juggling} with expressions invloving
mixed types. For example, \verb&2 + ``42''& is a valid expression that will
evaluate to $44$. Also, the type of any variable may change during the
execution.  For example, \verb&$a = 2; $a = new Foo();& is perfectly valid.

\subsubsection{Object Oriented Programing}
PHP supports OOP as of PHP4, but many features have later been added into PHP5.
The OO model as well as the syntax used is closely related to a subset of what
Java offers: public/protected/private visibility, single inheritance, with the
support of interfaces. PHP provides object properties, object methods, static
properties, static methods and class constants. You can dynamically define an
object property in PHP, its visibility will default to public. However, you
cannot dynamically define methods, static members, or constants.

\subsubsection{Limitation on analysis}
Due to its dynamicity, many PHP features will get in the way of a sound analysis.
Here is a non-exaustive list of such features:
\begin{itemize}
  \item \textbf{autoload}\footnote{PHP Autoload: http://php.net/autoload}:
    PHP allows you to trigger a function call in case an undefined class was
    used.  This function call is usually used to subsequently load the
    appropriate class at runtime. This feature allows programmers to only load
    classes on demand.
  \item \textbf{\_\_call}\footnote{PHP Overloading: http://php.net/\_\_call}:
    If a method call is defined in an object, calling an undefined method of
    that object will instead call the \verb&__call& method, with the name of
    the original method, and the arguments used.
  \item \textbf{\_\_callStatic}:
    This is similar to \verb&__call& but works for undefined static method
    calls.
  \item \textbf{\_\_get}\footnote{PHP Overloading: http://php.net/\_\_set}:
    This method will get called in case an access to an undefined object
    property is done. The value returned by the method will correspond to the
    property value.
  \item \textbf{ArrayAccess}\footnote{ArrayAccess Interface: http://php.net/arrayaccess}:
    The ArrayAccess interface allows an object to be used as an array. For example,
    if we have \verb&class Foo implements ArrayAccess&, then
    \verb&$foo = new Foo; echo $foo['index']& is accepted.
  \item \textbf{Dynamic accesses}
    PHP allows multiple ways to dynamically access variables, classes, members\ldots
    Here are a couple of examples:
    \begin{itemize}
      \item \verb/$$n/:
        Accesses the variable named by the value of \verb/$n/.
      \item \verb/$$$n/:
        Accesses the variable named by the value of \verb/$$n/.
      \item \verb/new $n()/:
        Constructs an object of the class named by the value of \verb/$n/.
      \item \verb/$c::$n()/:
        Call the static function named by the value of \verb/$n/ on the class
        named by the value of \verb/$n/.
      \item \verb/$n()/:
        Call the function named by the value of \verb/$n/.
    \end{itemize}
  \item \textbf{eval}\footnote{PHP Eval: http://php.net/eval}:
    The \verb&eval()& construct allows a developer to evaluate the code passed in as a string.
\end{itemize}

\subsection{Typeflow analysis}
The main focus of this analysis tool is on types. Even though PHP is dynamically typed,
and provide very few indications of types. It is still interresting to try to infer types
and check for type safety.

\subsection{Helpers}
In order to check that parts of the tool was working as expected, some helpers were
developped to retrieve information about each steps. The main helpers are able to generate:
\begin{itemize}
  \item A list of tokens generated from the file
  \item A dot file representing the Syntax Tree graph
  \item A dot file representing the Abstract Syntax Tree
  \item A dot file representing the Control Flow Graph, highlighting control structures
    such as \verb=if=, \verb=for=, \verb=foreach=, \verb=while=, \verb=dowhile=
\end{itemize}

Those helpers were used to generate the figures presented in this report.

\section{Limitations and future work}
As the timespan of this project was only one semester, only some limited
analysis has been done, and some shortcuts were taken.  For example, PHP
references were ignored as they add much more complexity to the process, while
their usage is limited. Additionally, even though the code base for the syntax
is PHP 5.3, some of its features such as namespaces or closures has been left
out. The aim of this project being to provide something useful, some implementation
details were ignored, which means that the current codebase contains several
\emph{TODO} annotation, that would have to be looked over whenever some
additional time on this project.

Much more analysis could be done, such as value ranges analysis, that can help
guide the traversal of the CFG to gain precision. Also, the model used for
arrays could be precised to allow more fine-grained analysis.

\section{Conclusions}
This project was really interessant because it was based on a real language, used
every day by millions of users wolrdwide. Such a tool, once reaching stability
and soundness to some extent can make a considerable impact. However, due to the
number of lacks in the current implementation, the tool as-is wouldn't be of
much help in most cases. I hope that I'll find more time to work on it, either
on my free time or as a school-related project, because I feel that there is a
real gap to fill regarding static analysis of PHP related programs and scripts.

\addcontentsline{toc}{section}{References}
\begin{thebibliography}{99}
  \bibitem{tigerbook}A.W. Appel {\it Modern Compiler Implementation in Java}
    Cambridge University Press (2002)
  \bibitem{recdescent}\mbox{http://lara.epfl.ch/dokuwiki/compilation:} \mbox{example\_efficient\_code\_for\_conditionals}
\end{thebibliography}


\end{document}
